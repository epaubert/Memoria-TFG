\chapter{Análisis del problema}

En esta sección justificaré las diferentes decisiones que he tomado durante este proyecto.

\section{Placa}

\begin{table}[h]
\centering
\renewcommand{\arraystretch}{1.2}
\begin{tabularx}{\textwidth}{|X|X|X|X|X|X|}
\hline
\textbf{} 			& \textbf{MC13224V} 	& \textbf{Raspberry Pi 4}	    & \textbf{STM32F7}			& \textbf{Arduino Mega 2560}    & \textbf{ESP32} 		 \\ \hline
\textbf{CPU} 			& ARM7TDMI 		& ARM Cortex-A72		    & ARM Cortex-M7   		    	& ATmega 2560			& Xtensa LX6 			 \\ \hline
\textbf{Frecuencia} 		& 24 MHz 		& 1.5 GHz			    & 216 MHz 	  		    	& 16 MHz		    	& 240 MHz			 \\ \hline
\textbf{Memoria Flash} 		& 128 KB 		& MicroSD (256 GB max.)		    & 2 MB 		  	    	& 256 KB 		    	& 4 MB				 \\ \hline
\textbf{RAM} 			& 96 KB 		& 1-8 GB LPDDR4 		    & 512 KB 	  		    	& 8 KB 			    	& 520 KB 			 \\ \hline
\textbf{Periféricos} 		& GPIO, UART, SPI, I2C 	& GPIO, UART, SPI, I2C, HDMI, USB   & GPIO, UART, SPI, I2C, DAC, ADC	& GPIO, UART, SPI, I2C 	    	& GPIO, UART, SPI, I2C, DAC, ADC \\ \hline
\textbf{Conexión} 		& ZigBee 		& Ethernet, WiFi, BT		    & Ethernet				& -- 			    	& WiFi, BT			 \\ \hline
\textbf{Consumo de Potencia} 	& 80 mW 		& 2.7 W (idle), 7.6 W (max) 	    & 0.3 W				& 0.5 W 		    	& 0.3 W (idle), 0.7 W (max) 	 \\ \hline
\textbf{Precio aprox.} 		& Descatalo-gada 	& 50 - 90€ 			    & 500€ 				& 50€			    	& 10€				 \\ \hline
\end{tabularx}
\caption{Comparativa de placas}
\label{Tab:placas}
\end{table}

La primera elección que influirá en el resto de decisiones posteriores es la de la placa en la que se desarrollará este proyecto. Hoy en día existen multitud de placas para sistemas empotrados, desde arduinos que rondan los 20-30€ a FPGAs en los miles de euros, cada una con sus características.

Al elegir una placa para un sistema empotrado no necesariamente queremos la que tiene mayor capacidad de cómputo, si no una que se ajuste a nuestras necesidades. Una placa más potente necesitará más energía para funcionar, que en caso de que el sistema funcione con una batería será peor elección que una más eficiente si no necesitamos tanta capacidad de procesamiento. Tecnologías como Bluetooth y Wifi pueden ser muy útiles para algunos proyectos, pero si no vamos a hacer uso de ellos son componentes que aumentan el precio y el tamaño de la placa innecesariamente.

En la tabla \ref{Tab:placas} he recopilado algunas placas relevantes para este proyecto. He añadido el Arduino Mega y el ESP32 a pesar de no estar basadas en ARM, aunque utilizan arquitecturas basadas en RISC [\ref{Def:RISC}].

% FIXME:

% TODO: Podría justificar la elección en que al ser FreeRTOS muy popular ya existen port para la mayoría de estas placas¿?

\subsection{Elección final}
Dado que este proyecto se plantea como una extensión de la asignatura de Sistemas Empotrados hay una placa que es de especial relevancia, la Redbee Econotag r3. A pesar de estar descatalogada, está disponible para retirar de la biblioteca de la Universidad de Granada, de forma que no supone un desembolso.

Es un kit de desarrollo para la placa MC13224V con una interfaz JTAG. Esto permite que se pueda desarrollar para esta placa y debuger utilizando OpenOCD y gdb.

\section{Sistemas operativos de Tiempo Real}

% FIXME: 
\begin{table}[h]
\centering
\renewcommand{\arraystretch}{1.5}
\begin{tabularx}{\textwidth}{|X|X|X|X|X|X|X|}
\hline
\textbf{Caracte-rísticas}	&  \textbf{Licencia}						& \textbf{Precio}		 & \textbf{Arquitec-turas}			& \textbf{Certifi-caciones}	& \textbf{Aplica-ciones} \\ \hline
\textbf{Deos}			&  \cellcolor{red!33}Propietario				& Precio no público		 & x86, ARM					& DO-178C			& Aviónica, defensa \\ \hline
\textbf{Free-RTOS}		&  \cellcolor{green!33}Software libre (MIT)			& Gratis			 & ARM, AVR, PIC, x86, RISC-V, etc.		& N/A				& IoT, sistemas de control industrial \\ \hline
\textbf{embOS}			&  \cellcolor{red!33}Propietario				& Desde \$3,000			 & ARM, Cortex-M, RISC-V, Renesas, MSP430	& N/A				& Automati-zación industrial, equipos médicos \\ \hline
\textbf{INTE-GRITY}		&  \cellcolor{red!33}Propietario				& Precio no público		 & ARM, x86, PowerPC, MIPS			& DO-178B, ISO 26262		& Automo-ción, sistemas críticos \\ \hline
\textbf{Keil RTX}		&  \cellcolor{red!33}Propietario				& Incluido con Keil MDK-ARM	 & ARM, Cortex-M				& N/A				& IoT, sistemas embebidos \\ \hline
\textbf{RTIC}			&  \cellcolor{green!33}Software libre (MIT /Apache 2.0)		& Gratis			 & ARM, Cortex-M				& N/A				& Dispositi-vos médicos, automatización industrial \\ \hline
\end{tabularx}
\label{tab:Comparativa de sistemas operativos de tiempo real}
\end{table}

\subsection{Propietarios}
Dada la naturaleza de este proyecto vamos a descartar los RTOSs [\ref{Def:Sistemas operativos de tiempo real}] propietarios, pero no sin antes hablar un poco de ellos para obtener una idea general del mercado de los RTOSs [\ref{Def:Sistemas operativos de tiempo real}].

\subsubsection{Deos}
Digital Engeneering Operative System (Deos), desarrollado por DCC-I, es un RTOS [\ref{Def:Sistemas operativos de tiempo real}] utilizado en campos como aviación, defensa y aeroespacial. Cumple con las certificación DO-178C para sofware de aviación. Utiliza una arquitectura de microkernel y tiene un alto grado de determinismo. También ofrece protección y particionamiento de memoria para mejorar la seguridad.

Posee un entorno de desarrollo integrado (IDE), soporte para actualizaciones en caliente y soporte a largo plazo. El precio no es público ya que depende de la aplicación y de los módulos que se vayan a utilizar.

\subsubsection{embOS}
Desarrollado por SEGGER, es un RTOS [\ref{Def:Sistemas operativos de tiempo real}] orientado a sistemas embebidos con los estándares de seguridad IEC 61508 SIL 3, IEC 62304 Class C (equipamiento médico), y ISO 26262 ASIL D (automovilismo).

Gratis para educación y evaluación, de pago para uso comercial.
\subsubsection{Integrity}
(Green Hills Software)
\subsubsection{Keil RTX}
(ARM)

\subsection{Libres}
\subsubsection{FreeRTOS}
(Amazon)
\subsubsection{RTIC}
https://rtic.rs/2/book/en/
\subsection{FreeRTOS}


% Pequeña sección sobre esto, pero corta sección sobre esto, pero corta
% PROBABLEMENTE ELIMINADA
% \section{Lenguaje}
% \subsection{C}
% \subsection{C++}
% \subsection{Rust}
% \subsection{Zig}
% \subsection{Carbon}
% \subsection{Elección}

