\chapter{Antecedentes}

% \noindent\fbox{
% 	\parbox{\textwidth}{
%     En este capítulo hablaré del contexto actual en el que se desarrolla este projecto y daré algunas definiciones relevantes para facilitar la lectura del proyecto.
% 	}
% }\\

% El software libre y sus licencias \cite{gplv3} ha permitido llevar a cabo una expansión del aprendizaje de la informática sin precedentes.

\section{Sistemas embebidos}
% NOTE: "ordenadores" o "sistemas informáticos"?
Cuando hablamos de ordenadores generalmente se piensa en computadores de sobremesa u  ordenadores portátiles, teléfonos inteligentes, tabletas o incluso servidores. A pesar de la ubicuidad de estos sistemas, muchos de los sistemas informáticos con los que interactuamos a diario son invisibles, pues están embebidos.

% \begin{center}\colorbox{cyan!10}{\fbox{ \begin{minipage}{11cm}
    Los Sistemas embebidos o empotrados, que abreviaré como \Def{SE}, son combinaciones de \emph{software} y \emph{hardware} (opcionalmente con partes mecánicas), de propósito específico y con un número reducido y definido de funciones. Pueden ser independientes o formar parte de otro sistema. Generalmente tienen una serie de restricciones como el uso de energía y capacidad para trabajar bajo condiciones adversas. Algunos ejemplos podrían ser el lector de tarjetas de metro o autobús, TV, \emph{router}, frigorífico, TPV, la multitud de sistemas que hay en un coche... \cite{es_glossary} \cite{marwedel}
% \end{minipage} }} \end{center}

% \begin{center}\colorbox{cyan!10}{\fbox{ \begin{minipage}{11cm}
    En cambio, los \textbf{ordenadores de propósito general} son una combinación de \emph{hardware} y \emph{software} diseñados para, valga la redundancia, un propósito general, es decir, sin un límite establecido de funciones. \cite{es_glossary}\\
% \end{minipage} }} \end{center}

Dos ejemplos claros de la diferencia entre un sistema empotrado y uno de propósito general serían la Nintendo Switch y la Steam Deck. La Nintendo Switch es un sistema empotrado, una consola diseñada para jugar a videojuegos que no tiene, por ejemplo, la función de navegar por Internet por motivos de seguridad. 
A pesar de su similaridad estética con la Switch, la Steam Deck, por contra, se podría considerar un ordenador de propósito general ya que, aunque su principal función es la ejecución de juegos, tiene todo un sistema \textit{Linux-Based} accesible por el usuario.


%La Steam Deck, por contra, es un sistema con un factor de forma similar pero es un ordenador de proposito general que no ha sido limitado por diseño, aunque su principal proposito sea el de jugar a videojuegos.

% TODO: Insertar imágenes de Switch y Steam Deck aqui
%
%
%
%
%

%  \begin{center} \colorbox{yellow!10} { \fbox{ \begin{minipage}{11cm}
%     Existe un debate sobre si un teléfono o una tablet es un sistema empotrado, y con respecto a los \emph{dumbphones} o móviles no inteligentes es cierto dado que tienen un reducido número de funciones, pero mi opinión es que los móviles actuales se pueden utilizar como ordenadores de propósito general y la línea entre tablet y portátil se difumina cada vez más.
% \end{minipage} } } \end{center}

En el centro de todo sistema empotrado hay un microprocesador, pero para hablar más en detalle de los microprocesadores primero debemos hablar de las ISAs.

\subsection{ISA}
Una arquitectura de conjunto de instrucciones, \Def{ISA} por sus siglas en inglés (Instruction Set Arquitecture), es una especificación formal que define las instrucciones que puede ejecutar un procesador y como influirán el el \emph{hardware}. Abarca tanto la codificación, los tipos y longitud de datos, los registros disponibles y su formato, los modos de ejecución y los modos de acceso a memoria. Algunas ISAs consisten de un núcleo al que se le pueden añadir extensiones dependiendo de las capacidades que queremos que tenga el microprocesador.\\

En la actualidad hay dos enfoques en el desarrollo de ISAs. Estos son CISC y RISC. Si bien la inmensa mayoría de los sistemas embebidos utilizan arquitecturas RISC, también veremos las características de CISC para tener más contexto.

% FIXME:
\subsection{CISC}
Un computador con conjunto de instrucciones complejas o \Def{CISC} por sus siglas en ingles (Complex Instruction Set Computer) es una ISA [\ref{Def:ISA}] cuyo objetivo es reducir el número de instrucciones que ejecuta cada programa. Sus principales características son:
\begin{itemize}
    \item Un conjunto de instrucciones extenso.
    \item Instrucciones complejas que pueden utilizar varios ciclos de reloj.
    \item Instrucciones de longitud variable.
    \item Instrucciones aritméticas y lógicas que pueden operar directamente sobre la memoria RAM.
\end{itemize}

La familia de arquitecturas CISC más popular se conoce como x86-64, AMD64 o simplemente x64. Sobre ella ejercen un duopolio de facto Intel y AMD, existe la posibilidad de adquirir una licencia pero son extremadamente costosas y no muy útiles sin los recursos ni la experiencia de los ingenieros de estas empresas.

% FIXME:
\subsection{RISC}
Un computador con un conjunto de instrucciones reducido \Def{RISC}, por sus siglas en inglés (Reduced Instruction Set Computer) es una ISA [\ref{Def:ISA}] que se centra en la eficiencia de cada operación realizada por el procesador. Las principales características de este enfoque son:
\begin{itemize}
    \item Un conjunto de instrucciones reducido.
    \item Instrucciones simples que se ejecutan en un ciclo de reloj.
    \item Todas las instrucciones tienen la misma longitud
    \item Sólo las instrucciones de carga y almacenamiento acceden a la memoria.
\end{itemize}
Estas características minimizan la fase de decodificación de la instrucción y hacen que el procesador esté menos tiempo decodificando instrucciones y más tiempo ejecutándolas. Otro beneficio es que permite la creación de procesadores de menor tamaño y más eficientes, lo que los hace ideales para SEs [\ref{Def:SE}]. 

La familia de arquitecturas RISC más extendida es ARM, aunque también se está popularizado el uso de RISC-V.

\subsection{ARM}
ARM se refiere tanto a la familia de arquitecturas como a la empresa que las diseña.

\Def{ARM} es una familia de arquitecturas RISC cuya primera versión (ARMv1) fue lanzada en 1985 y la más moderna (ARMv9) fué lanzada en 2021. Son ampliamente utilizadas en \emph{smartphones}, tabletas, sistemas embebidos y más recientemente en ordenadores portátiles como algunos Chromebooks, Apple Silicon y los portátiles Windows con procesadores Snapdragon X.\\

La empresa Arm Holdings plc no fabrica los procesadores, sino que se dedica a licenciar tanto la arquitectura para que otras empresas diseñen procesadores (Apple Silicon), como los diseños de procesadores creados por la propia empresa(Procesadores Snapdragon de Qualcomm). También proporcionan herramientas de desarrollo de \emph{software} para sus procesadores.

\subsection{RISC-V}
Dada la relevancia del \emph{software} libre en el contexto de este trabajo es importante destacar la ISA [\ref{Def:ISA}] \Def{RISC-V}. Es una arquitectura RISC de \emph{hardware} libre, es decir, cuyo diseño es accesible, modificable y comercializable sin necesidad de licencia.

Comenzó como un proyecto de la Universidad de Berkley (California) en 2010 y hoy en día es usado ampliamente utilizado en la industria. La fundación que supervisa el proyecto es apoyada por empresas como AMD, Nvidia, Oracle, Google y Qualcom. \cite{RISC-V_BOOK}

\subsection{GPIO}
Entrada y salida de propósito general, \Def{GPIO} por sus siglas en inglés \emph{General Purpouse Input Output}. Se refiere a una serie de pines muy comunes en circuitos integrados que pueden programarse para una variedad de funciones a través del microcontrolador. 
%\cite{GPIO_BIB}

\subsection{UART}
\emph{Universal Asynchronous Receiver-Transmitter} por sus siglas en inglés, \Def{UART} es un protocolo de comunicación por cable que se usa entre dos computadores, generalmente un circuito integrado y un ordenador. \cite{UART_BIB}

\subsection{HAL}
Capa de abstracción de \emph{hardware} o \emph{Hardware Abstraction Layer} en inglés, un \Def{HAL} es una capa de código que conecta el \emph{software} que hay por encima con el \emph{hardware} que hay justo debajo. Cada HAL es específica a un \emph{hardware} concreto pero permite al desarrollador del \emph{software} abstraerse del \emph{hardware}. \cite{Yoo2003}

\subsection{Driver}
Un manejador o \Def{driver} es un \emph{software} que actúa de intermediario entre un \emph{hardware} específico y un \emph{software} o un sistema operativo para que se pueda hacer un uso eficiente y seguro de ese \emph{hardware}.

\subsection{BSP}
Un Board Support Package o \Def{BSP} en el contexto de los sistemas embebidos es un \emph{firmware} que actúa como una interfaz estándar entre un \emph{hardware} concreto y un kérnel o programa que corre sobre él. Proporciona una secuencia de arranque y \emph{drivers} para los dispositivos de la placa, entre otras funciones.\cite{fredericksfaq}\\

\section{Computación en Tiempo Real}
Los sistemas de \Def{Tiempo Real} son un tipo de sistema donde el funcionamiento correcto del mismo no depende solamente de que el resultado lógico sea correcto, sino también del tiempo requerido para llegar a ese resultado \cite{stankovic}. Esto no quiere decir necesariamente que las tareas deban tardar el mínimo tiempo posible, sino que el tiempo que se tarda en realizar una tarea debe de estar acotado, ser predecible y consistente.

Un ejemplo de cómo la consistencia se valora más que la velocidad es que, en muchos sistemas de tiempo real, no se utiliza memoria caché, ya que a pesar de que puede acelerar la computación, los aciertos y fallos de caché son difíciles de predecir y pueden provocar inconsistencias en el tiempo de ejecución \cite{MILLIGAN1996}.

La computación en tiempo real puede aplicarse a ordenadores, sistemas embebidos y redes de comunicaciones. Algunos ejemplos pueden ser sistemas de control de procesos, sistemas de conducción autónoma y sistemas de control de tráfico aéreo.\\

% TODO:
Antes de hablar de los tipos de sistemas de tiempo real hay que definir lo que son las\Def{Deadlines}. En el contexto de los sistemas de tiempo real, este concepto se refiere al tiempo límite que tiene una tarea para completarse sin que se considere un error.

\subsection{Tiempo real «duro» y «blando»}
Cuando hablamos de sistemas de tiempo real existe la distinción entre sistemas de tiempo real «duros» (\emph{hard real time}) y «blandos» (\emph{soft real time}). \emph{Hard real time} se refiere a sistemas críticos donde una \emph{deadline} [\ref{Def:Deadlines}] no cumplida puede tener resultados catastróficos, como puede ser en campos como la medicina o la aeroespacial. En cambio el \emph{soft real time} abarca problemas donde las \emph{deadline} [\ref{Def:Deadlines}] no son tan estrictas e incumplirlas degradará la experiencia, pero el sistema podrá seguir funcionando. Ejemplos de esto son el \emph{streaming} de elementos multimedia o la comunicación utilizando \emph{VoIP (Voice over IP)}. Mientras que el tiempo real «blando» se suele implementar dentro de otros sistemas que no requieren tiempo real, el tiempo real «duro» se implementa dentro de sistemas especializados.

%Dos ejemplos que muestran esta distinción son el ABS de un coche y una videoconsola. El ABS es un sistema de tiempo real «duro» ya que si no reacciona a tiempo puede provocar una colisión. En cambio una videoconsola es un sistema de tiempo real blando, debe realizar los cálculos necesarios para cada \emph{frame} del juego a tiempo de manera que se mantenga un \emph{framerate} estable, pero si se retrasa alguno se puede seguir jugando al juego.\\

Esta clasificación de «duro» y «blando» no es estricta, sino que es más un espectro. Existen sistemas con mayores y menores tolerancias, pero no existe una linea definida que separe claramente los sistemas de tiempo real «duros» y «blandos». Aun así la mayoría del trabajo de investigación que se realiza en torno al tiempo real se refiere a la parte del espectro que se considera tiempo real «duro».

También en este espectro existe el \emph{firm real time} que se refiere a sistemas donde, si no se alcanza una \emph{deadline} [\ref{Def:Deadlines}], el resultado de la operación es irrelevante. Un ejemplo es un sistema que predice el tiempo, si el cálculo de la simulación es que va a llover pero ya ha empezado a llover el resultado no sirve para nada \cite{wang2017rtes}.

\subsection{Sistemas Operativos de Tiempo Real}
Los sistemas de tiempo real «duro» suelen utilizar sistemas operativos especializados llamados sistemas operativos de tiempo real, \Def{RTOS} por sus siglas en inglés (Real Time Operative System).

La diferencia más importante de los sistemas operativos de tiempo real con respecto a los de propósito general es la planificación. Estos últimos utilizan \emph{FIFO}, \emph{Round-Robin} o colas con prioridad entre otros (o una combinación de varios) para planificar las tareas que el sistema o los usuarios van creando y terminando.

Los sistemas de tiempo real tienen una filosofía muy distinta, ya que el sistema va a tener un número muy reducido de funciones y tareas en comparación a un sistema operativo tradicional, pero también cada tarea tendrá un tiempo límite en el que debe completarse o \emph{deadline} [\ref{Def:Deadlines}]. Por ello, la planificación debe calcularse en el peor caso (con el máximo número de tareas que se dará en la realidad), ya sea manualmente o con un algoritmo, para que todas las tareas terminen dentro de su \emph{deadline} [\ref{Def:Deadlines}].

Algunos de los sistemas operativos de tiempo real más utilizados son Deos, embOS, FreeRTOS, Integrity y Keil RTX \cite{lynx2024rtos}.

\section{La intersección de los sistemas embebidos y los sistemas de tiempo real}
Los sistemas embebidos [\ref{Def:SE}] y de tiempo real han ido de la mano desde sus orígenes. Uno de los primeros ejemplos es el AGC.

El Computador de Navegación del Apollo, AGC por sus siglas en inglés Apollo Guidance Computer, se considera uno de los primeros sistemas embebidos de la historia. Fue utilizado en las misiones Apollo para controlar la orientación y navegación de los módulos de mando y lunares. Y utilizaba un RTOS [\ref{Def:RTOS}] que permitía ejecutar hasta 8 tareas a la vez y iba rotando entre estas tareas según la prioridad de cada una.

Otro ejemplo más moderno, además de obras de ingeniería tan impresionantes en la automoción o aeroespacial, es un monitor de bebés  utilizando Keil RTX para comunicar una cámara y un monitor \cite{LLLBabyMonitor}.\\

% %FIXME:
% FIXME:
% Hoy en día la inmensa mayoría de los RTOSs [\ref{Def:RTOS}] se usan en sistemas embebidos [\ref{Def:SE}]...
%
% bla bla bla el diagrama de Venn es casi un círculo bla bla bla...

