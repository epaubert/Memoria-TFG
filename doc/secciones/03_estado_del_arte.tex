\chapter{Antecedentes}

% \noindent\fbox{
% 	\parbox{\textwidth}{
%     En este capítulo se tratará el problema que se va a intentar abordar en este trabajo, junto con una descripción del mismo, y los objetivos y restricciones para su solución.
% 	}
% }\\


El software libre y sus licencias \cite{gplv3} ha permitido llevar a cabo una expansión del
aprendizaje de la informática sin precedentes.

\section{Descripción de la tecnología}
\subsection{Sistemas embebidos}
% NOTE: "sistemas informáticos" o "ordenadores"?
Cuando hablamos de ordenadores generalmente se piensa sobremesas o portátiles, y tal vez en teléfonos inteligentes, tablets o tal vez incluso servidores. A pesar de la ubicuidad de estos sistemas, muchos de los ordenadores con los que interactuamos a diario son invisibles, estan embebidos.

\begin{center}\colorbox{cyan!10}{\fbox{ \begin{minipage}{11cm}
Los \textbf{sistemas embebdidos} son combinaiones de software y hardware (opcionalmente con partes mecánicas), de proposito específico y con un número reducido y definido de funciones. Pueden ser independientes o formar parte de otro sistema. Generalmente tienen una serie de restricciones como el uso de energía y capacidad para trabajar bajo condiciones adversas. Ejemplos: Lector de tarjetas de metro o autobús, TV, router, frigorífico, TPV, la multitud de sistemas que hay en un coche...
\end{minipage} }} \end{center}

\begin{center}\colorbox{cyan!10}{\fbox{ \begin{minipage}{11cm}
En cambio, los \textbf{ordenadores de propósito general} es una combinación de hardware y software diseñados para un proposito general, es decir, sin un límite establecido de funiciones. Ejemplos: PC, teléfono inteligente, servidor...
\end{minipage} }} \end{center}

Dos ejemplos claros de la diferencia entre un sistema empotrado y uno de proposito general serían la Nintendo Switch y la Steam Deck. La Nintendo Switch es un sistema empotrado, una consola diseñada para jugar a videojuegos que no tiene, por ejemplo, la función de navegar por internet por motivos de seguridad. La Steam Deck es un sistema con un factor de forma similar pero es un ordenador de proposito general que no ha sido limitado por diseño, aunque su principal proposito sea el de jugar a videojuegos.

\begin{center}\colorbox{yellow!10}{\fbox{ \begin{minipage}{11cm}
Hay quien argumenta que un teléfono o una tablet es un sistema empotrado, y con respecto a los \say{dumbphones} o móviles no inteligentes es cierto dado que tienen un reducido número de funciones, pero los móviles actuales son ordenadores de propósito general y la línea entre tablet y portátil se difumina cada vez más.
\end{minipage} }} \end{center}

\subsection{Sistemas de Tiempo Real}

% \usepackage{dirtytalk}
\say{Los sistemas de tiempo real son un tipo de sistema donde el funcionamiento correcto del mismo no depende solamente de que el resultado lógico sea correcto, si no también del tiempo requerido para llegar a ese resultado.} [Stankovic, 1988].

\section{Historia}

\section{Aplicaciones}
