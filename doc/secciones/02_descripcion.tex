\chapter{Descripción del problema}
\section{El problemas}
En la actualidad existen multitud de placas diseñadas para sistemas embebidos, así como una gran cantidad de sistemas operativos y firmwares que se utilizan para facilitar el desarrollo de aplicaciones para las mismas.
Para muchas de estas aplicaciones modernas es útil o necesario el uso de sistemas de tiempo real, pues permite establecer restricciones de tiempo predecibles y consistentes, además de poder crear una planificación que se ajuste a nuestros objetivos.\\

El problema radica en la inmensa cantidad de placas que, aun teniendo los requisitos necesarios para el proyecto que tengamos entre manos, carecen de la posibilidad de utilizar un RTOS, perdiendo flexibilidad y aumentando los tiempos de desarrollo, y obligándonos a buscar opciones que no sean tan favorables.
Además los sistemas operativos de tiempo real prediminantemente utilizados en la industria son desarrollados por empresas que poseen software propietario, lo que dificulta el aprendizaje y el acceso a estos sistemas.

% En la actualidad, el uso de sistemas embebidos esta creciendo de manera exponencial en nuestra sociedad, así como la implementación de sistemas operativos de tiempo real (RTOS). Si bien estos últimos estan implementados en multitud de placas de sistemas empotrados, aún existe un gran número de placas que no tienen posibilidad de utilizar este RTOS, perdiendo la flexibilidad y la funcionalidad que estos podrian ofrecer.

\section{La solución}
Se propone la realización de una adaptación o \textit{port} de un sistema operativo de tiempo real a una placa ARM los siguientes objetivos:
\begin{itemize}
	\item Profundizar en el conocimiento sobre la arquitectura de los sistemas operativos, en especial los de tiempo real.
	\item Profundizar en el concimiento de los sistemas embebidos
	\item Profundizar en el conocimiento sobre arquitectura ARM.
	\item Aprender sobre implementación de las restricciones de el tiempo real.
\end{itemize}
