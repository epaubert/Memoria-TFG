\chapter{Descripción del problema}
\section{El problema}
En la actualidad existen multitud de placas diseñadas para sistemas embebidos y una gran cantidad de sistemas operativos y firmwares que se utilizan para facilitar el desarrollo de aplicaciones para las mismas.
Para muchas de estas aplicaciones modernas es útil o necesario el uso de sistemas de tiempo real ya que permite establecer restricciones de tiempo predecibles y consistentes.
Aun así hay una inmensa cantidad de placas que carecen de la posibilidad de utilizar un RTOS, perdiendo flexibilidad y aumentando los tiempos de desarrollo.
Además los sistemas operativos de tiempo real prediminantemente utilizados en la industria son propietarios, lo que dificulta el aprendizaje sobre estos sistemas.

% En la actualidad, el uso de sistemas embebidos esta creciendo de manera exponencial en nuestra sociedad, así como la implementación de sistemas operativos de tiempo real (RTOS). Si bien estos últimos estan implementados en multitud de placas de sistemas empotrados, aún existe un gran número de placas que no tienen posibilidad de utilizar este RTOS, perdiendo la flexibilidad y la funcionalidad que estos podrian ofrecer.

\section{La solución}
Se propone la realización de una adaptación o \textit{port} de un sistema operativo de tiempo real a una placa ARM con el objetivo de aprender sobre la arquitectura de los sistemas operativos, las restricciones de el tiempo real 

\itemize
