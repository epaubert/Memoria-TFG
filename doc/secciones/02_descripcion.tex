\chapter{Descripción del problema}
\section{El problema}
En la actualidad existen multitud de placas diseñadas para sistemas embebidos [\ref{Def:SE}], así como una gran cantidad de sistemas operativos y \emph{firmwares} que se utilizan para facilitar el desarrollo de aplicaciones para las mismas.

Para muchas de estas aplicaciones modernas es útil o necesario el uso de sistemas de tiempo real, pues permite establecer restricciones de tiempo predecibles y consistentes, además de poder crear una planificación que se ajuste a nuestros objetivos.\\

El principal problema radica en la inmensa cantidad de placas que, aun teniendo los requisitos necesarios para el proyecto que tengamos entre manos, carecen de la posibilidad de utilizar un RTOS, perdiendo flexibilidad y aumentando los tiempos de desarrollo, y obligándonos a buscar opciones que no sean tan favorables.
Además, los sistemas operativos de tiempo real utilizados más ampliamente en la industria son desarrollados por empresas que los desarrollan como \emph{software} propietario, lo que dificulta el aprendizaje y el acceso a estos sistemas.

% En la actualidad, el uso de sistemas embebidos esta creciendo de manera exponencial en nuestra sociedad, así como la implementación de sistemas operativos de tiempo real (RTOS). Si bien estos últimos están implementados en multitud de placas de sistemas empotrados, aún existe un gran número de placas que no tienen posibilidad de utilizar este RTOS, perdiendo la flexibilidad y la funcionalidad que estos podrían ofrecer.

\section{Solución propuesta}
Se propone la realización de una adaptación o \emph{port} de un sistema operativo de tiempo real a una placa ARM [\ref{Def:ARM}] que no posea dicha compatibilidad, así como la realización de un programa \emph{demo} que demuestre su funcionalidad.

% Con el objetivo de promover el aprendizaje sobre estos temas se priorizará el uso del \emph{software} libre y componentes de bajo coste.
% \begin{itemize}
% 	\item Profundizar en el conocimiento sobre la arquitectura de los sistemas operativos, en especial los de tiempo real.
% 	\item Profundizar en el concimiento de los sistemas embebidos
% 	\item Profundizar en el conocimiento sobre arquitectura ARM.
% 	\item Aprender sobre implementación de las restricciones de el tiempo real.
% \end{itemize}

\section{Restricciones}
\begin{itemize}
    \item \textbf{FOSS:} Se priorizará el uso de \emph{software} libres y de código abierto en todo momento.
    \item \textbf{Presupuesto:} Se utilizarán componentes \emph{hardware} accesibles o de bajo coste.
    \item \textbf{Tiempo:} El tiempo límite para este proyecto es la convocatoria extraordinaria de presentación de trabajos de fin de grado, en septiembre de 2024.
\end{itemize}

% TODO:
\section{Objetivos}
\subsection{Objetivos de Investigación}
\begin{itemize}
	% Sistemas empotrados
    \item \textbf{O-I.1} Investigar sobre sistemas embebidos.
    \item \textbf{O-I.2} Investigar distintas placas para sistemas embebidos.
    \item \textbf{O-I.3} Investigar las capacidades y compatibilidad de la placa que escojamos.
	% Tiempo real
    \item \textbf{O-I.4} Investigar sobre sistemas de tiempo real.
    \item \textbf{O-I.5} Investigar distintos RTOSs [\ref{Def:RTOS}].
    \item \textbf{O-I.6} Investigar la estructura  y requerimientos de distintos RTOS [\ref{Def:RTOS}].
	% Combinados
    \item \textbf{O-I.7} Investigar sobre el uso de tiempo real en sistemas embebidos.
\end{itemize}


\subsection{Objetivos de Diseño}
\begin{itemize}
    \item \textbf{O-D.1} Diseño de actores y casos de uso. % TODO: ???
    \item \textbf{O-D.2} Extensión, si lo fuese necesaria, de los \emph{drivers} o BSP [\ref{Def:BSP}] de la placa escogida.
    \item \textbf{O-D.3} Diseño y programación del port del RTOS [\ref{Def:RTOS}] escogido a la placa
    \item \textbf{O-D.4} Diseño y programación del una demo que permita comprobar el funcionamiento del port.
\end{itemize}
