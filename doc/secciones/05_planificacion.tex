\chapter{Planificación}

\section{Etapas}

Voy a dividir la planificación en tres etapas, aprendizaje, desarrollo y depuración. Cada una de estas fases contendrán pasos que he seguido en la implementación asociados a ellos.

\subsection{Etapa 1: Aprendizaje}
La primera fase consiste en la recopilación de información y el aprendizaje necesario para realizar el proyecto.

\begin{itemize}
\item Familiarizarse con la estructura de FreeRTOS
\item (Re)familiarizarse con el BSP
\end{itemize}

\subsection{Etapa 2: Desarrollo}
\begin{itemize}
\item Creación de los archivos necesarios para el port
\item Integración del BSP
\item Preparación de los CMakeFiles y primera compilación
\item Ampliación del BSP
\item Modificación de los archivos del port
\end{itemize}

\subsection{Etapa 3: Depuración}
\begin{itemize}
\item Demo básica
\item Demo ampliada
\end{itemize}

\section{Temporización}
Esta temporización supone una semana de trabajo de 20 horas. De media se realizarán cuatro horas de trabajo al día cinco días a la semana.

\begin{table}[h!]
\centering
\begin{tabularx}{\textwidth}{|X|c|c|c|}
\hline
\textbf{Tarea} & \textbf{Tiempo estimado} & \textbf{Fecha inicio} & \textbf{Fecha fin} \\ \hline
Familiarizarse con la estructura de FreeRTOS & 1 semana & 15/05/2024 & 21/05/2024 \\ \hline
(Re)familiarizarse con el BSP & 1 semana & 29/05/2024 & 04/06/2024 \\ \hline
Creación de los archivos necesarios para el port & 1 semana & 22/05/2024 & 28/05/2024 \\ \hline
Integración del BSP & 1.5 semanas & 05/06/2024 & 14/06/2024 \\ \hline
Preparación de los CMakeFiles y primera compilación & 1.5 semanas & 17/06/2024 & 26/06/2024 \\ \hline
Ampliación del BSP & 1.5 semanas & 27/06/2024 & 08/07/2024 \\ \hline
Modificación de los archivos del port & 1 semana & 09/07/2024 & 15/07/2024 \\ \hline
Demo básica & 1 semana & 16/07/2024 & 22/07/2024 \\ \hline
Demo ampliada & 1.5 semanas & 23/07/2024 & 1/08/2024 \\ \hline
\end{tabularx}
\caption{Plan de trabajo para el proyecto.}
\label{tabla:plan_trabajo}
\end{table}

\begin{figure}[!ht]
\centering
\includegraphics[width=\textheight, angle=270]{img/Temporización-TFG.png}
\caption{Diagrama de Gantt de la temporización}
\label{fig:Gantt_1}
\end{figure}

\section{Presupuesto}
\emph{WITTENSTEIN High Integrity Systems} ofrece como servicio la realización de un port de FreeRTOS a una plataforma que no lo soporte, así que contacté con ellos para saber cual sería el presupuesto de un proyecto similar a este pero no recibí respuesta.

\subsection{Sueldo}
Según lo comentado por mi tutor y lo observado en diversas ofertas de trabajo por Internet, el sueldo usual de un ingeniero de sistemas embebidos \emph{junior} ronda los 20.000€ al año. Convertido a salario por hora, esto son 9.62€/hora. La duración prevista del proyecto son una 11 semanas, trabajando 20 horas semanales. $11 * 20 = 220$ horas totales. La remuneración total sería $220 * 9,62 = 2116,40$€.

\subsection{Costes de software y hardware}
Todo el \emph{software} utilizado ha sido gratuito y de código abierto, por lo que no se ha incurrido en ningún coste de este tipo.

En términos de \emph{hardware}, el único que he acabado utilizando ha sido la Econotag, que ha sido tomado prestado de la biblioteca de la UGR.


\subsection{Coste final}
Por lo tanto, el presupuesto total ha sido de $2116,40$€, completamente en concepto de sueldo.

% \section{Seguimiento del desarrollo}
%
% como a día 1 de agosto no había conseguido que funcionase correctamente, pasé a trabajar en la memoria. Finalmente, la útima semana de agosto continué el trabajo en el proyecto y conseguí que funcionara
