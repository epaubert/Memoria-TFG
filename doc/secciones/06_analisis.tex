\chapter{Análisis del problema}

En esta sección justificaré las diferentes decisiones que he tomado durante este proyecto.

\section{Placa}

\begin{table}[h]
\centering
\renewcommand{\arraystretch}{1.2}
\begin{tabularx}{\textwidth}{|X|X|X|X|X|X|}
\hline
\textbf{} 			& \textbf{MC13224V} 	& \textbf{Raspberry Pi 4}	    & \textbf{STM32F7}			& \textbf{Arduino Mega 2560}    & \textbf{ESP32} 		 \\ \hline
\textbf{CPU} 			& ARM7TDMI -S 		& ARM Cortex-A72		    & ARM Cortex-M7   		    	& ATmega 2560			& Xtensa LX6 			 \\ \hline
\textbf{Frecuencia} 		& 24 MHz 		& 1.5 GHz			    & 216 MHz 	  		    	& 16 MHz		    	& 240 MHz			 \\ \hline
\textbf{Memoria Flash} 		& 128 KB 		& MicroSD (256 GB max.)		    & 2 MB 		  	    	& 256 KB 		    	& 4 MB				 \\ \hline
\textbf{RAM} 			& 96 KB 		& 1-8 GB LPDDR4 		    & 512 KB 	  		    	& 8 KB 			    	& 520 KB 			 \\ \hline
\textbf{Periféricos} 		& GPIO, UART, SPI, I2C 	& GPIO, UART, SPI, I2C, HDMI, USB   & GPIO, UART, SPI, I2C, DAC, ADC	& GPIO, UART, SPI, I2C 	    	& GPIO, UART, SPI, I2C, DAC, ADC \\ \hline
\textbf{Conexión} 		& ZigBee 		& Ethernet, WiFi, BT		    & Ethernet				& -- 			    	& WiFi, BT			 \\ \hline
\textbf{Consumo de Potencia} 	& 80 mW 		& 2.7 W (idle), 7.6 W (max) 	    & 0.3 W				& 0.5 W 		    	& 0.3 W (idle), 0.7 W (max) 	 \\ \hline
\textbf{Precio aprox.} 		& Descatalo-gada 	& 50 - 90€ 			    & 500€ 				& 50€			    	& 10€				 \\ \hline
\end{tabularx}
\caption{Comparativa de placas}
\label{Tab:placas}
\end{table}

Si bien es cierto que la placa a elegir está determinada por los requisitos no funcionales, daremos un repaso al mercado actual con objeto de tener algo de contexto.\\

La primera elección que influirá en el resto de decisiones en un proyecto de un sistema empotrado es la placa en la que se desarrollará este proyecto. Hoy en día existen multitud de placas para sistemas embebidos, desde sistemas Arduino que rondan los 20-30€, a FPGAs en el marco de los miles de euros, cada una con sus características.

Al elegir una placa para un sistema empotrado no necesariamente queremos la que tiene mayor capacidad de cómputo, sino una que se ajuste a nuestras necesidades. Una placa más potente necesitará más energía para funcionar, que en caso de que el sistema funcione con una batería, será peor elección que una más eficiente si no necesitamos tanta capacidad de procesamiento. Tecnologías como Bluetooth y Wifi pueden ser muy útiles para algunos proyectos pero, si no vamos a hacer uso de ellas, son componentes que aumentan el precio y el tamaño de la placa innecesariamente.

En la tabla \ref{Tab:placas} he recopilado algunas placas relevantes para este proyecto. He añadido el Arduino Mega y el ESP32 a pesar de no estar basadas en ARM, aunque utilizan arquitecturas basadas en RISC [\ref{Def:RISC}].

% FIXME:

% TODO: Podría justificar la elección en que al ser FreeRTOS muy popular ya existen port para la mayoría de estas placas¿?

\subsection{Redbee Econotag r3}
Dado que este proyecto se plantea como una extensión de la asignatura de Sistemas Empotrados, utilizaremos la Redbee Econotag r3. A pesar de estar descatalogada, está disponible para retirar de la biblioteca de la Universidad de Granada, de forma que no supone un desembolso.

Es un kit de desarrollo para la placa MC13224V con una interfaz JTAG. Esto permite que se pueda desarrollar para esta placa y debugear utilizando OpenOCD y gdb.

\section{Sistemas operativos de Tiempo Real}

Al igual que la elección de la placa, la elección de RTOS viene dada por los requisitos funcionales, pero resulta relevante hacer una recopilación de RTOSs del mercado actual.

% FIXME: 
\begin{table}[h]
\centering
\renewcommand{\arraystretch}{1.5}
\begin{tabularx}{\textwidth}{|X|X|X|X|X|X|X|}
\hline
\textbf{Caracte-rísticas}	&  \textbf{Licencia}						& \textbf{Precio}		 & \textbf{Arquitec-turas}			& \textbf{Certifi-caciones}	& \textbf{Aplica-ciones} \\ \hline
\textbf{Deos}			&  \cellcolor{red!33}Propietario				& Precio no público		 & x86, ARM					& DO-178C			& Aviónica, defensa \\ \hline
\textbf{Free-RTOS}		&  \cellcolor{green!33}Software libre (MIT)			& Gratis			 & ARM, AVR, PIC, x86, RISC-V, etc.		& N/A				& IoT, sistemas de control industrial \\ \hline
\textbf{embOS}			&  \cellcolor{red!33}Propietario				& Desde \$3,000			 & ARM, Cortex-M, RISC-V, Renesas, MSP430	& N/A				& Automati-zación industrial, equipos médicos \\ \hline
\textbf{INTE-GRITY}		&  \cellcolor{red!33}Propietario				& Precio no público		 & ARM, x86, PowerPC, MIPS			& DO-178B, ISO 26262		& Automo-ción, sistemas críticos \\ \hline
\textbf{Keil RTX}		&  \cellcolor{red!33}Propietario				& Incluido con Keil MDK-ARM	 & ARM, Cortex-M				& N/A				& IoT, sistemas embebidos \\ \hline
\textbf{RTIC}			&  \cellcolor{green!33}Software libre (MIT /Apache 2.0)		& Gratis			 & ARM, Cortex-M				& N/A				& Dispositi-vos médicos, automatización industrial \\ \hline
\end{tabularx}
\label{tab:Comparativa de sistemas operativos de tiempo real}
\end{table}

\subsection{Propietarios}
Dada la naturaleza de este proyecto vamos a descartar los RTOSs [\ref{Def:RTOS}] propietarios, pero no sin antes hablar un poco de ellos para obtener una idea general del mercado de los RTOSs [\ref{Def:RTOS}].

\subsubsection{Deos}
Digital Engeneering Operative System (Deos), desarrollado por DCC-I, es un RTOS [\ref{Def:RTOS}] utilizado en campos como aviación, defensa y aeroespacial. Cumple con las certificación DO-178C para sofware de aviación. Utiliza una arquitectura de microkernel y tiene un alto grado de determinismo. También ofrece protección y particionamiento de memoria para mejorar la seguridad.

Posee un entorno de desarrollo integrado (IDE), soporte para actualizaciones en caliente y soporte a largo plazo. El precio no es público ya que depende de la aplicación y de los módulos que se vayan a utilizar.

\subsubsection{embOS}
Desarrollado por SEGGER, es un RTOS [\ref{Def:RTOS}] orientado a sistemas embebidos [\ref{Def:SE}] con los estándares de seguridad IEC 61508 SIL 3, IEC 62304 Class C (equipamiento médico), y ISO 26262 ASIL D (automovilismo).

Gratis para educación y evaluación, de pago para uso comercial.

\subsubsection{INTEGRITY}
Desarrollado por Green Hills Software, es un RTOS con certificado POSIX. Soporta múltiples arquitecturas, entre ellas ARM, x86, PowerPC y MIPS.

En su versión INTEGRITY-178B, certificada para software aeronáutico, se usa en varios aviones militares y en el avión comercial A380 de Airbus.

%TODO:
(Insertar imángen del A380)

\subsubsection{Keil RTX}
Desarrollado por Keil, una subsidiaria de ARM (la empresa), para procesadores ARM [\ref{Def:ARM}] (la arquitectura) y Cortex-M. Esto hace que esté especialmente optimizado para estas arquitecturas. Está integrado en el Keil MDK (Microcontroller Development Kit), por lo que es gratis para usos no comerciales.

\subsection{Libres}
\subsubsection{FreeRTOS}
Desarrollado desde 2003 por Richard Barry, que se unió a Amazon en 2017, haciendo que FreeRTOS pasase a ser parte del software desarrollado por AWS y cambiando de una licencia GPLv2 modificada a MIT.

Es de los RTOS más utilizados si no el que más (los datos no son concluyentes) en el contexto de los sistemas empotrados. Al ser de código abierto y tener más de 20 años de desarrollo la comunidad al rededeor de este sistema operativo es enorme y soporta una gran cantidad de arquitecturas.

%TODO:
\textbf{OpenRTOS} y \textbf{SafeRTOS} son versiones comerciales de FreeRTOS distribuidas por WITTENSTEIN high integrity systems

\subsubsection{RTIC}
Son las siglas de \emph{Real-Time Interrupt-driven Concurrency}. Es un proyecto escrito en el lenguaje de programación Rust que utiliza las fortalezas de este lenguaje en el contexto de los sistemas de tiempo real.

En la documentación \cite{RTIC} se comenta que definirlo como un RTOS [\ref{Def:RTOS}] depende de los antecedentes y la perspectiva y el de cada uno. Para los desarrolladores es un RTOS que en lugar de utilizar un gestor de tareas por software utiliza hardware específico (NVIC en procesadores Cortex-M, CLIC en procesadores RISC-V...) para la planificación de tareas.

\subsection{Elección}
FreeRTOS

% Pequeña sección sobre esto, pero corta sección sobre esto, pero corta
% PROBABLEMENTE ELIMINADA
% \section{Lenguaje}
% De un tiempo a esta parte han aparecido una serie de lenguajes de programación con el objetivo de reemplazar a C. En un principio estaba considerando realizar este port en uno de esos lenguajes, ya que varios de ellos puede interactuar con C de forma nativa. Al final decidí que iba a ser una inversión de esfuerzo muy grande por muy poca recompensa, pero voy a hacer un pequeño resumen de algunos de estos lenguajes en el contexto de los sistemas empotrados.
% \subsection{C}
% \subsection{C++}
% \subsection{Rust}
% \subsection{Zig}
% \subsection{Odin}
% \subsection{Carbon}
