\chapter{Introducción}

Este proyecto es \emph{software} libre, y está liberado con la licencia \cite{gplv3}.

\section{Motivación}

La motivación de este Trabajo de Fin de Grado surgió tras cursar la asignatura de Sistemas Empotrados del grado. Sentí que, aunque habíamos trabajado muy bien la base de los sistemas con el mismo nombre, quería profundizar algo más porque es un tema que considero muy interesante. 

Además, durante los años de la carrera, había oído hablar de los sistemas operativos de tiempo real, que también llamaron mucho mi atención, de manera que he decidido trabajar estos dos intereses en este proyecto.  

\section{Estructura}
\begin{enumerate}
    \item \textbf{Introducción:} En este capítulo se aborda la motivación del proyecto y se describe la estructura del documento.
    \item \textbf{Descripción del problema:} Se detallan el objetivo del proyecto, las restricciones a considerar y los requisitos necesarios para llevarlo a cabo.
    \item \textbf{Antecedentes:} Se presenta el contexto tecnológico, definiciones y datos relevantes que permiten comprender el entorno de los sistemas empotrados y de tiempo real.
    \item \textbf{Estudio de requisitos:} En este apartado se describen los casos de uso, los requisitos y los actores que definen y limitan el alcance del proyecto.
    \item \textbf{Análisis del problema:} Se analizan las decisiones relacionadas con la elección de \emph{hardware} y \emph{software} para asegurar el correcto funcionamiento del sistema.
    \item \textbf{Planificación:} Se establece un cronograma para el proyecto y se discuten otros factores importantes como el presupuesto y las fases de implementación del mismo.
    \item \textbf{Implementación:} Se detalla el proceso de implementación del proyecto, siguiendo las fases definidas en la planificación y abordando los problemas encontrados en las diferentes etapas.
    \item \textbf{Conclusiones y trabajos futuros:} Este capítulo final ofrece una valoración del trabajo realizado, detalla los conocimientos adquiridos durante la realización de este proyecto y propone posibles líneas de trabajo futuras.
    \item \textbf{Apéndices:} Aquí se incluye un glosario, una explicación más detallada de la implementación y algunos extractos de código.
\end{enumerate}
